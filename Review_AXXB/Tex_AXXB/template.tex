\documentclass{llncs}
\usepackage{graphicx}
%\usepackage{lineno}
\usepackage{xspace}
%\usepackage{amsthm}
\usepackage{url}
\usepackage{hyperref}

\usepackage{epsfig} %% for loading postscript figures
\usepackage{amssymb,amsmath}
\usepackage{amsfonts}
\usepackage{multirow}
%\usepackage{natbib}
\usepackage{relsize}
\usepackage{graphicx}
\usepackage{color}
\usepackage{comment}
%\usepackage{algorithm2e}


%\newtheorem{theorem}{Theorem}[section]
%\newtheorem{corollary}{Corollary}
%\newtheorem*{main}{Main Theorem}
%\newtheorem{lemma}[theorem]{Lemma}
%\newtheorem{proposition}{Proposition}
%\newtheorem{conjecture}{Conjecture}
%\newtheorem*{problem}{Problem}
%\theoremstyle{definition}
%\newtheorem{definition}[theorem]{Definition}
%\newtheorem{remark}{Remark}
%\newtheorem*{notation}{Notation}

\providecommand{\keywords}[1]{\textbf{\textit{Index terms---}} #1}
\newcommand{\uu}{{\bf u}}
\newcommand{\vv}{{\bf v}}
\newcommand{\ww}{{\bf w}}
\newcommand{\WW}{{\bf W}}
\newcommand{\nn}{{\bf n}}
\newcommand{\ttt}{{\bf t}}
\newcommand{\pp}{{\bf p}}
\newcommand{\qq}{{\bf q}}
\newcommand{\aaa}{{\bf a}}
\newcommand{\bbb}{{\bf b}}
\newcommand{\yy}{{\bf y}}
\newcommand{\xx}{{\bf x}}
\newcommand{\OO}{\mathbb{O}}
\newcommand{\II}{\mathbb{I}}
\newcommand{\IR}{\mathbb{R}}
\newcommand{\IZ}{\mathbb{Z}}
\newcommand{\half}{\frac{1}{2}}
\newcommand{\bea}{\begin{eqnarray*}}
\newcommand{\eea}{\end{eqnarray*}}
\newcommand{\bfmu}{\mbox{\boldmath $\mu$ \unboldmath} \hskip -0.05 true in}
\newcommand{\bfphi}{\mbox{\boldmath $\phi$ \unboldmath} \hskip -0.05 true in}
\newcommand{\beq}{\begin{equation}}
\newcommand{\eeq}{\end{equation}}
\newcommand{\bfomega}{\mbox{\boldmath $\omega$ \unboldmath} \hskip -0.05 true in}
\newcommand{\om}{\omega}
\newcommand{\Om}{\Omega}
\newcommand{\bom}{\bfomega}
\def\tab{ {\hskip 0.15 true in} }
 \def\vtab{ {\vskip 0.1 true in} }
 \def\htab{ {\hskip 0.1 true in} }
  \def\ntab{ {\hskip -0.1 true in} }
 \def\vtabb{ {\vskip 0.0 true in} }

\newcommand\defeq{\stackrel{\mathrm{def}}{=}}
% Add a period to the end of an abbreviation unless there's one
% already, then \xspace.
\makeatletter
\DeclareRobustCommand\onedot{\futurelet\@let@token\@onedot}
\def\@onedot{\ifx\@let@token.\else.\null\fi\xspace}

\newcommand{\yasu}[1]{\textcolor{red}{{[Yasu: #1]}}}

\def\eg{\emph{e.g}\onedot} \def\Eg{\emph{E.g}\onedot}
\def\ie{\emph{i.e}\onedot} \def\Ie{\emph{I.e}\onedot}
\def\cf{\emph{c.f}\onedot} \def\Cf{\emph{C.f}\onedot}
\def\etc{\emph{etc}\onedot} \def\vs{\emph{vs}\onedot} 
\def\wrt{w.r.t\onedot} \def\dof{d.o.f\onedot}
\def\etal{\emph{et al}\onedot}

\newcommand{\figcaption}[1]{\def\@captype{figure}\caption{#1}}
\newcommand{\tblcaption}[1]{\def\@captype{table}\caption{#1}}
\makeatother

\newcommand{\argmax}{\operatornamewithlimits{argmax}}
\newcommand{\argmin}{\operatornamewithlimits{argmin}}


\def\title#1{{\noindent\Large{\bf #1}\par}}
\def\author#1{\begin{center}{\sc #1\par}\end{center}}
\renewcommand{\section}[1]{\vspace{0.1in}\noindent{\large\bf{#1}}\par\vspace{.05in}\par\nopagebreak}

\makeatletter
\def\thebibliography#1{\section{Recommended Readings\@mkboth
{REFERENCES}{REFERENCES}}\list
{[\arabic{enumi}]}{\settowidth\labelwidth{[#1]}\leftmargin\labelwidth
\advance\leftmargin\labelsep
\usecounter{enumi}}
\def\newblock{\hskip .11em plus .33em minus .07em}
\sloppy\clubpenalty4000\widowpenalty4000
\sfcode`\.=1000\relax}
\makeatother


\begin{document}
\pagestyle{empty}

\title{Chapter X : Review of Methods for Solving AX=XB Sensor Calibration Problem}
\author{\textbf{Qianli Ma},  \textbf{Gregory S. Chirikjian}\\
   {
	Robot and Protein Kinematics Laboratory\\
	Laboratory for Computational Sensing and Robotics\\
	Department of Mechanical Engineering\\
	The Johns Hopkins University\\
	Baltimore, Maryland, 21218\\
    Email: \{mqianli1, gchirik1\}@jhu.edu
    }	
}

%%%%%%%%%%%%%%%%%%%%%%%%%%%%%%%%%%%%%%%%%%%%%%%%%%%%%%%%%%%%%%%%%%%%%%
\begin{abstract}
{\it
An often used formulation of sensor calibration in robotics and computer vision is ``AX=XB'', where $A$, $X$, and $B$ are rigid-body motions with $A$ and $B$ given from sensor measurements, and $X$ is the unknown calibration parameter. Many methods have been proposed to solve $X$ given data streams of $A$ and $B$ under different scenarios. This chapter presents the most complete picture of the $AX=XB$ solvers up to date.  Firstly, a brief overview of the various important sensor calibration techniques is given and problems of interest are highlighted. Then a detailed review on the various ``AX=XB'' research algorithms is presented. The notations of the selected algorithms are unified to the largest extent in order to show the interconnections between the selected methods in a straightforward  way. Next, the criterion for data selection and various error metrics are introduced, which are of critical importance for evaluating the performance of $AX=XB$ solvers. After that, numerical comparison are performed among the most important algorithms. At the end, both the advantages and disadvantages of the reviewed methods are summarized.
}
\end{abstract}

\keywords{Sensor Calibration, Hand-Eye Calibration, Humanoid Robot, Review}

%%%%%%%%%%%%%%%%%%%%%%%%%%%%%%%%%%%%%%%%%%%%%%%%%%%%%%%%%%%%%%%%%%%%%%

\section{nomenclature}
\begin{itemize}

\item{$SO(3)$}{$SO(3) \doteq \{R|RR^{T} = R^{T}R = \mathbb{I}$ and $\det(R) = 1$ where $R \in \mathbb{R}^{3 \times 3}\}$}
\item{$SE(3)$}{$SE(3) \doteq \{H|H=\left( R \;   \ttt ; {\bf 0}^{T} 1 \right) \in \mathbb{R}^{4 \time 4},$ where $R \in SO(3)$ and $\ttt \in \mathbb{R}^{3}\}$ }
\item{$so(3)$}{$so(3) \doteq \{\Omega |R = exp(\Omega)$, where $\Omega \in \mathbb{R}^{3 \times 3}$ and $R \in SO(3)$ $\}$}
\item{$se(3)$}{$se(3) \doteq \{\Xi |H = exp(\Xi)$, where $\Xi =\left(\Omega \; \xi; \; {\bf 0}^{T} 0 \right)$ and $\Omega \in SO(3), \xi \in \mathbb{R}^{3}, H \in SE(3) \}$}
\item{$exp()$}{The matrix exponential of a square matrix. }
\item{$log()$}{The matrix logarithm of a square matrix}
\item{$H$}{A general rigid body transformation ($H \in SE(3)$) }
\item{$\mathfrak{H}$}{ If $H$ is an element of a Lie Group, $\mathfrak{H}$ is the corresponding element in the Lie algebra}
\item{$\mathbb{O}_n$}{$n \times n$ zero matrix}
\item{$\mathbb{I}_n$}{$n \times n$ identity matrix}
\item{$\{E_i\}$}{the set of ``natural" basis elements for Lie algebra}
\item{$^\vee$}{For Lie algebra, the ``vee" operator is defined such that $\left(\displaystyle\sum\limits_{i=1}^n x_iE_i\right)^{\vee}\doteq(x_1,x_2,...,x_n)^T$ where $n = 3$ for $so(3)$ and $n = 6$ for $se(3)$}
\item{$\hat{}$}{For Lie algebra, the ``hat" operator is the inverse of the ``vee" operator. $\widehat{(x_1,x_2,...,x_n)^T} \doteq \left( \displaystyle\sum\limits_{i=1}^n x_iE_i \right)$}
\item{$\circ$}{The operator defined for group product}
\item{$\odot$}{The operator defined for quaternion product}
\item{$\hat{\odot}$}{The operator defined for dual quaternion product}
\item{$\otimes$}{The operator defined for Kronecker product}
\item{$vec$}{The ``vec'' operator is defined such that $vec(A) = [a_{11},...,a_{1m},a_{21},...,a_{2m},...,a_{n1},...,a_{nm}]^{T}$ for $A = [a_{ij}] \in \mathbb{R}^{m \times n}$}
\item{${\bf p}$}{For $P\in G$ (where $G$ is a Lie group, e.g. $SE(3)$ or $SO(3)$), ${\bf p}=(p_1,p_2,...,p_n)^T=\log^{\vee}(P)$}
\item{$A_i$}{A rigid body transformation ($A_i \in SE(3)$), associated with one sensor measurement source}
\item{$B_i$}{A rigid body transformation ($B_i \in SE(3)$), usually assocated with one sensor measurement source}
\item{$X$}{The rigid body transformation ($X_i \in SE(3)$) that relates $A_i$ to $B_i$}
\item{$R_{H}$}{The rotation matrix of any general transformation matrix $H \in SE(3)$}
\item{$\ttt_{H}$}{The translation vector of any general transformation matrix $H \in SE(3)$}
\item{$\nn_{H}$}{The axis of rotation for $R_{H}$}
\item{$\tilde{\nn}$}{The skew-symmetric representation of the axis of rotation ($\nn_{H}$)}
\item{$\theta_{H}$}{The angle of rotation for $R_{H}$ about $\nn_{H}$}
\item{$\rho$}{A probability distribution of $H\in G$ on SE(3)}
\item{$M$}{For $\rho$, $M$ is the mean of the distribution}
\item{$\Sigma$}{For $\rho$, $\Sigma$ is the covariance of the distribution about the mean, $M$}
\item{$Ad$}{For the Lie group $G$ and the Lie algebra $\mathfrak{G}$, the adjoint operator is the transformation\\ Ad: $G \rightarrow$ GL($\mathfrak{G}$), defined as\\ Ad($H_1$)$\mathfrak{H}_2 \doteq \frac{d}{dt}(H_1 \circ e^{t \mathfrak{H}_2} \circ H_1^{-1})$}
\item{$Ad(H) $}{The adjoint matrix with columns $(H E_i H^{-1})^{\vee}$}
\end{itemize}



%%%%%%%%%%%%%%%%%%%%%%%%%%%%%%%%%%%%%%%%%%%%%%%%%%%%%%%%%%%%%%%%%%%%%%%%%%%%%%
\section{Synonyms}
\begin{itemize}
\item{Synonym 1}
\item{Synonym 2}
\item{...}
\end{itemize}
List only acronyms and those words with exactly the same meaning. All 
words that appear as synonyms will be directly linked back to the home 
entry, with no additional information given.

\section{Related Concepts}
\begin{itemize}
\item{Related concept 1}
\item{Related concept 2}
\item{...}
\end{itemize}

This includes keywords. All related concepts will point to other entries in the encyclopedia. 


\section{Definition}

One or two sentences summarizing the concept of the entry.

\section{Background}

Describe the background of the entry.

\section{Theory or Application, or Both}

Describe the theory (application) of the entry. It can be either 
Theory or Application, or it could be both of them.

\section{Open problems (optional)}

Describe the new trends, unsolved problems related to the entry.

\section{Experimental Results (optional)}

Experimental results that help the understandings of the entry. It 
can be videos, codes, etc.

\nocite{*}
\bibliographystyle{plain}
\bibliography{template}
10-20 citations to important literature.
These references will be hyperlinked to the original source material. 
\end{document}

