\documentclass{llncs}
\usepackage{graphicx}
%\usepackage{lineno}
\usepackage{color}
\usepackage{amsmath}
\usepackage{amssymb}
\usepackage{xspace}
%\usepackage{amsthm}
\usepackage{url}
\usepackage{hyperref}

\newcommand\defeq{\stackrel{\mathrm{def}}{=}}
% Add a period to the end of an abbreviation unless there's one
% already, then \xspace.
\makeatletter
\DeclareRobustCommand\onedot{\futurelet\@let@token\@onedot}
\def\@onedot{\ifx\@let@token.\else.\null\fi\xspace}

\newcommand{\yasu}[1]{\textcolor{red}{{[Yasu: #1]}}}

\def\eg{\emph{e.g}\onedot} \def\Eg{\emph{E.g}\onedot}
\def\ie{\emph{i.e}\onedot} \def\Ie{\emph{I.e}\onedot}
\def\cf{\emph{c.f}\onedot} \def\Cf{\emph{C.f}\onedot}
\def\etc{\emph{etc}\onedot} \def\vs{\emph{vs}\onedot} 
\def\wrt{w.r.t\onedot} \def\dof{d.o.f\onedot}
\def\etal{\emph{et al}\onedot}

\newcommand{\figcaption}[1]{\def\@captype{figure}\caption{#1}}
\newcommand{\tblcaption}[1]{\def\@captype{table}\caption{#1}}
\makeatother

\newcommand{\argmax}{\operatornamewithlimits{argmax}}
\newcommand{\argmin}{\operatornamewithlimits{argmin}}

%\newtheorem{prop}{Proposition}
%\newtheorem{prf}{Proof}

\def\title#1{{\noindent\Large{\bf #1}\par}}
\def\author#1{\begin{center}{\sc #1\par}\end{center}}
\renewcommand{\section}[1]{\vspace{0.1in}\noindent{\large\bf{#1}}\par\vspace{.05in}\par\nopagebreak}

\makeatletter
\def\thebibliography#1{\section{Recommended Readings\@mkboth
{REFERENCES}{REFERENCES}}\list
{[\arabic{enumi}]}{\settowidth\labelwidth{[#1]}\leftmargin\labelwidth
\advance\leftmargin\labelsep
\usecounter{enumi}}
\def\newblock{\hskip .11em plus .33em minus .07em}
\sloppy\clubpenalty4000\widowpenalty4000
\sfcode`\.=1000\relax}
\makeatother


\begin{document}
\pagestyle{empty}

\title{Title of the entry}
\author{Author, Affiliation}

\section{Synonyms}
\begin{itemize}
\item{Synonym 1}
\item{Synonym 2}
\item{...}
\end{itemize}
List only acronyms and those words with exactly the same meaning. All 
words that appear as synonyms will be directly linked back to the home 
entry, with no additional information given.

\section{Related Concepts}
\begin{itemize}
\item{Related concept 1}
\item{Related concept 2}
\item{...}
\end{itemize}

This includes keywords. All related concepts will point to other entries in the encyclopedia. 


\section{Definition}

One or two sentences summarizing the concept of the entry.

\section{Background}

Describe the background of the entry.

\section{Theory or Application, or Both}

Describe the theory (application) of the entry. It can be either 
Theory or Application, or it could be both of them.

\section{Open problems (optional)}

Describe the new trends, unsolved problems related to the entry.

\section{Experimental Results (optional)}

Experimental results that help the understandings of the entry. It 
can be videos, codes, etc.

\nocite{*}
\bibliographystyle{plain}
\bibliography{template}
10-20 citations to important literature.
These references will be hyperlinked to the original source material. 
\end{document}

